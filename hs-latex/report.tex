\documentclass[11pt,a4paper, halfparskip]{scrartcl}
\usepackage[utf8]{inputenc}
\usepackage[ngerman]{babel}
\usepackage[T1]{fontenc}
\usepackage{amsmath}
\usepackage{amsfonts}
\usepackage{amssymb}
\usepackage{graphicx}
\usepackage[bookmarksnumbered,pdftitle={Name}]{hyperref}
\usepackage[left=2cm,right=2cm,top=2cm,bottom=2cm]{geometry}
\usepackage{lmodern}

\usepackage{geometry}
\geometry{a4paper, top=25mm, left=40mm, right=40mm, bottom=30mm,
headsep=10mm, footskip=12mm}

\title{Bericht zum Laborprojekt}
\author{Jeremy Puchta \and Jonathan Lange \and Ali Al-Ali}
\renewcommand*{\titlepagestyle}{empty}

\begin{document}

\begin{titlepage}
	\centering
	\includegraphics[width=1\textwidth]{Unilogo}\par\vspace{1cm}
	{\scshape\LARGE Universität Leipzig \par}
	\vspace{0.5cm}
	{\scshape\Large Abteilung Automatische Sprachverarbeitung\par}
	\vspace{0.2cm}
	{\scshape\large Fortgeschrittene Methoden des Information Retrieval\par}
	\vspace{1cm}
	{\huge\bfseries Bericht zum Laborprojekt\par}
	\vspace{1cm}
	{\Large Jeremy Puchta -- Jonathan Lange -- Ali Al-Ali \par}

	\vfill
	
	{\large \today\par}
\end{titlepage}

\newpage

\thispagestyle{empty}
\tableofcontents

\newpage
\pagenumbering{arabic}
\section{Einleitung}

Der vorliegende Bericht erläutert den Aufbau sowie die Funktionsweise der Suchmaschine \textbf{Historical News Search}, welche im Rahmen des Laborprojektes innerhalb des Moduls \textit{Fortgeschrittene Methoden des Information Retrieval} im Wintersemester 2018 / 2019 erstellt wurde. 
Ziel des Laborprojektes war es, die vermittelten Vorlesungsinhalte zu vertiefen und diese bei der Erstellung einer domänenspezifischen Suchmaschine umzusetzen. 
Die vorgestellte Suchmaschine dient der Exploration von historischen Nachrichten.
Verschiedene Szenarien sind als Anwendungsfälle für eine solche Suchmaschine denkbar. 
Ein Nutzer kann nach der Berichterstattung zu Personen, Gruppen, Orten oder bestimmten historischen Ereignissen suchen.
Als Beispiele für Suchanfragen können hierfür \texttt{reichstagswahl 1930}, \texttt{max schmeling} oder \texttt{nsdap} genannt werden.
Eine weitere Möglichkeit der Nutzung stellt die Ahnenforschung dar, bei welcher ein Nutzer in Zeitungsberichten nach Informationen über seine Vorfahren suchen kann und somit mehr über seine Familienhistorie erfahren kann.

Im folgenden Kapitel wird zunächst näher auf den Datensatz eingegangen und dieser mit Methoden der desktriptiven Statistik analysiert.
Die Architektur und der Aufbau der Suchmaschine werden in Kapitel 3 näher dargestllt. 
Dazu zählt insbesondere die Erstellung des Indizes sowie die Umsetzung des Suchprozesses über alle beteiligten Komponenten des Systems.
In Kapitel 4 werden zunächst die Evaluationsmethodik sowie die Resultate der Evaluation erläutert.
Weiterhin wird dargestellt, welche Anpassungen vorgenommen wurden, um die Effektivität der Suchmaschine zu verbessern.
Abschließend erfolgt in Kapitel 5 eine Zusammenfassung des Projektes und es wird ein Ausblick in Bezug auf eine Weiterentwicklung der Suchmaschine geliefert.

\section{Datenakquise und -analyse}

Die \textit{Staatsbibliothek zu Berlin} besitzt ein breites Spektrum von historisch bedeutsamen digitalisierten Zeitungen.
Diese werden im hauseigenen Zeitungsinformationssystem namens \textit{ZEFYS} kostenfrei bereitgestellt.
Die Digitalisate, Volltexte und Metadaten der \textit{Berliner Volks-Zeitung (BVZ)} dienen als Korpus für die im Rahmen des Laborprojektes entwickelte Suchmaschine.
Bei der Berliner Volks-Zeitung handelt es sich um eine von 1904 bis 1944 veröffentlichte regionale deutsche Tageszeitung aus Berlin.
Sie besitzt große Bedeutung für die Forschung im Bereich der Kulturwissenschaften, da sie im Gegensatz zu den meisten linken Parteizeitungen, über ein gutes Feuilleton verfügt. [QUELLE]
Der Zeitraum der Digitalisate beläuft sich auf die Jahre 1890 bis 1930, wobei einige Jahre stärker abgedeckt sind als andere.
Die Datensets umfassen jeweils strukturierte Metadaten im METS-XML-Containerformat für jede Ausgabe, per OCR erzeugte Volltexte im ALTO-XML-Format mit Wortkoordinaten, binarisierte TIFFs als Grundlage der OCR sowie JPEG2000-Bilder für die Anzeige.
Insgesamt umfasst der Datensatz 103.771 digitalisierte Seiten.

\subsection{Datenverarbeitung}
// GROBE ERKLÄRUNG DATENFORMAT 

// BESCHREIBUNG VERARBEITUNG DER XML DATEN ZU JSON - AUF PROBLEME AUFMERKSAM MACHEN

\subsection{Textstatistiken}
// TEXTSTATISTIKEN (DESKRIPTIVE STATISTIK) – INTERESSANTE VARIABLEN MARKIEREN

\section{Architektur}

In diesem Kapitel wird zunächst der Technologiestack des Systems beschrieben. 
Außerdem erfolgt die detaillierte Vorstellung des User Interfaces mit Erklärung der einzelnen Sichten bevor abschließend der Indizierungs- und der Suchprozess erläutert werden.
Abbildung 1 visualisiert die einzelnen Komponenten des Systems mittels UML-Komponentendiagramm. 

// UML-KOMPONENTENDIAGRAMM ARCHITEKTUR DER SUCHMASCHINE 

\subsection{Technologiestack}

Zur Umsetzung des Laborprojektes wird die Open-Source-Suchmaschine \textit{Elasticsearch} verwendet. 
\textit{Elasticsearch} ist in Java geschrieben und basiert auf der Java-Bibliothek \textit{Lucene}. 
Außerdem ist \textit{Elasticsearch} gebaut für den Einsatz auf verteilten Systemen und die Echtzeitverarbeitung von großen Datenmengen, was unter anderem zur Volltextsuche eingesetzt wird und \textit{Elasticsearch} somit zu einer ausgezeichneten Option für die Umsetzung einer Dokumentensuchmaschine macht.
\textit{Elasticsearch} stellt sämtliche Funktionen über eine programmiersprachenunabhängige REST-Schnittstelle zur Verfügung.
Um die Dokumente durchsuchbar zu machen, legt \textit{Elasticsearch} die Datenstruktur des \textit{Invertierten Index} an, in welcher für jeden Term gespeichert wird in welchem Dokument dieser auftritt.

Im Backend des Systems kommt das Python-Framework \textit{Flask} zum Einsatz. 
Bei \textit{Flask} handelt es sich um ein Mikroframework, welches eine einfache, jedoch ebenfalls erweiterbare und robuste Möglichkeit bietet APIs zu erstellen.
Im Frontend wird mit \textit{Angular} ein Framework verwendet, welches Entwickler durch sein Komponentensystem darin unterstützt modularen Quellcode zu produzieren, der sich einfach warten lässt.
Außerdem bringt es bereits viele Funktionen, die häufig im Frontend zum Einsatz kommen, wie zum Beispiel Routing mit. 
Zur Vereinfachung des Deployments und zur Optimierung des Entwicklungsprozesses kommt die Containervirtualisierungs – Technologie \textit{Docker} zum Einsatz.

// MUSS NOCH AN IRGENDETWAS GEDACHT WERDEN?

\subsection{Indizierungsprozess}



\subsection{Suchprozess}

\newpage
\section{Evaluation}

\newpage
\section{Zusammenfassung und Ausblick}

% ----------------------------------------------------------------------------

\newpage
\begin{footnotesize}

% IF YOU DO NOT USE BIBTEX, USE THE FOLLOWING SAMPLE SCHEME FOR THE REFERENCES
% ----------------------------------------------------------------------------

\begin{thebibliography}{99}

% For articles
%\bibitem{1}David Caswell, Konstantin Dörr, "Automated Journalism 2.0: Event-driven naratives", \emph{Journalism Practice}, 09.05.2017.

%\bibitem{2}Andreas Graefe, "Guide to Automated Journalism", \emph{Columbia University Academic Commons}, 07.01.2016.

%\bibitem{3}L. Leppänen, M. Munezero, M. Granroth-Wilding, H. Toivonen, "Data-Driven News Generation for Automated Journalism", \emph{University of Helsinki}, 2017

%\bibitem{4}Retresco GmbH, ``Retresco - Textgenerierung'', \emph{\\https://www.retresco.de/textgenerierung/}, zuletzt abgerufen 01.07.2018

%\bibitem{5}DFL Deutsche Fußball Liga GmbH, ``Schalke 04 krönt Saison mit Heimsieg - Eintracht Frankfurt verpasst vorerst Europa'', \emph{https://www.bundesliga.com/de/bundesliga/news/fc-schalke-04-eintracht-frankfurt-34-spieltag-spielbericht.jsp}, zuletzt abgerufen 01.07.2018

%\bibitem{6}Statistik zum Spiel FC Bayern München - Vfb Stuttgart auf Bundesliga.de, \\
%\emph{https://www.bundesliga.com/de/bundesliga/spieltag/2017-2018/34-fc\_bayern\_muenchen-vfb\_stuttgart.jsp?statistics}, zuletzt abgerufen 24.08.2018

\end{thebibliography}
%
%\section{Abbildungsverzeichnis}
%


% IF YOU USE BIBTEX,
% - DELETE THE TEXT BETWEEN THE TWO ABOVE DASHED LINES
% - UNCOMMENT THE NEXT TWO LINES AND REPLACE 'Name_Of_Your_BibFile'

%\bibliographystyle{unsrt}
%\bibliography{Name_Of_Your_BibFile}

\end{footnotesize}

\end{document}